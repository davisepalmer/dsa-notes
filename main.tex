\documentclass{tufte-book}

\title{Role of Algorithms in Computing}

\author[Role of Algorithms]{notes by dep}


%\geometry{showframe} % display margins for debugging page layout
\nobibliography{true}
\usepackage{minted}
\usepackage{graphicx} % allow embedded images
  \setkeys{Gin}{width=\linewidth,totalheight=\textheight,keepaspectratio} % set image keys if you're using graphicx
  \graphicspath{{images/}} % set of paths to search for images
\usepackage{amsmath}  % extended mathematics
\usepackage{booktabs} % book-quality tables
\usepackage{units}    % non-stacked fractions and better unit spacing
\usepackage{multicol} % multiple column layout facilities
\usepackage{lipsum}   % filler text
\usepackage{fancyvrb} % extended verbatim environments
  \fvset{fontsize=\normalsize}% default font size for fancy-verbatim environments

% Standardize command font styles and environments
\newcommand{\doccmd}[1]{\texttt{\textbackslash#1}}% command name -- adds backslash automatically
\newcommand{\docopt}[1]{\ensuremath{\langle}\textrm{\textit{#1}}\ensuremath{\rangle}}% optional command argument
\newcommand{\docarg}[1]{\textrm{\textit{#1}}}% (required) command argument
\newcommand{\docenv}[1]{\textsf{#1}}% environment name
\newcommand{\docpkg}[1]{\texttt{#1}}% package name
\newcommand{\doccls}[1]{\texttt{#1}}% document class name
\newcommand{\docclsopt}[1]{\texttt{#1}}% document class option name
\newenvironment{docspec}{\begin{quote}\noindent}{\end{quote}}% command specification environment

\begin{document}

% \maketitle % this prints the handout title, author, and date


%\printclassoptions

% Paragraph to start

% \section{This is a section}\label{sec:Section 1}
% \subsection{this is a subsection}\label{sec:subsection}
% I changed this now i changed it again now I am changing it again

% \sidenote{This is a side note.}

% \section{Testing out minted}
% \begin{minted}{cpp}

% int main() {
%   cout << "Hello World" << endl;
%   return 0;
% }
% \end{minted}

\section{Role of Algorithms}
\subsection{General Vocab}
\quad An \underline{algorithm} is a computational procedure that takes input and produces an output. It is a sequential action of computational steps that transforms input to output. An example of this in use would be a sorting problem. We are given an input of \textit{unsorted} values and we need to produce an ouput of sorted values; for this we use an algorithm.\sidenote{\underline{Permutation} - synonym for reordering in this case} A \textit{correct algorithm} is said to be one if for every input case\sidenote{\textit{Case} can also be called an instance} it returns the correct output. A \underline{data structure} is a way to store and organize data in order to facilitate access and modifications. There are a variety of different data structures as one cannot fulfill all types of problems. 
\\
\newthought{There is two main things when thinking of algorithms: \textit{efficiency} and \textit{memory}}.  






\end{document}
